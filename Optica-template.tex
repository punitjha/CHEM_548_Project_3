\documentclass[9pt,twocolumn,twoside]{optica}
\setboolean{shortarticle}{false}
\setboolean{minireview}{false}
\usepackage{mhchem}
\title{The H$\ddot{\text u}$ckel Molecular Orbital Theory And Its Extensions. }

\author[1,*]{Punit K. Jha}

\affil[1]{The University of Illinois at Urbana-Champaign}

\affil[*]{Corresponding author: punit2@illinois.edu}

\dates{Compiled \today}

\ociscodes{}

\doi{}

\begin{abstract}

The H$\ddot{\text u}$ckel method or the H$\ddot{\text u}$ckel molecular orbital method was proposed by Erich H$\ddot{\text u}$ckel in 1930. In this method, linear combination of atomic orbitals molecular orbitals (LCAO MO) is used for the determination of energies of molecular orbitals of $\pi$ electrons in planar conjugated hydrocarbon systems, such as ethene, benzene etc. This method forms the theoretical basis for H$\ddot{\text u}$ckel rule, which estimates whether a planar ring molecule will have aromatic properties or not. This theory only includes the $\pi$ electron molecular orbitals and neglects the sigma electrons since  the $\pi$ electrons determine the general properties of planar conjugated systems. In this project we develop a C++ program to:\\
1. Analyze the photoelectron spectra of polycyclic aromatic hydrocarbon (PAH),\\
2. Study the one dimensional $\pi$ band structure of polyacetylene and single-wall carbon nanotubes (SWNT),\\
3. Analyze the two-dimensional  $\pi$ band structure of graphene.\\
4. Use the Su-Schrieffer-Heeger Model to see the effect of bond-lenght variation on the energy bands and total energy of Benzene, Benzene Cation and 1,3,5,7-cyclooctatetraene (COTE), and show  the origins of the Jahn-Teller distortion (Extended Huckel method has been used in case of COTE ).\\
5. Use the Su-Schrieffer-Heeger Model to see the effect of bond-length variation on the energy bands and total energy of polyacetylene and discuss its effects with respect to the Peierls theorem.
\end{abstract}

\setboolean{displaycopyright}{false}

\begin{document}

\maketitle
\thispagestyle{fancy}
\ifthenelse{\boolean{shortarticle}}{\abscontent}{}

\section{Introduction}

The vast majority of polyatomic molecules can be thought of as consisting of a collection of 2-electron bonds between pairs of atoms. The picture of a $\sigma$ and $\pi$-bonding and anti-bonding orbitals for a molecule like CO\textsubscript{2} can be carried over to give a qualitative starting point for describing, say for instance, a  \ce{C\bond{=} O} in acetone. This picture is extremely useful in dealing with conjugated systems - that is, molecules that contain a series of alternating double/single bonds in their Lewis structure like $trans$-polyacetylene as shown in Figure \ref{fig:trans-poly}.

A common way to represent the electronic structure of $trans$-polyacetylene is by using either the 2 mesomeric Lewis structures represented in Figure \ref{fig:trans-poly2} (a) and (b) or the average structure of Figure \ref{fig:trans-poly2} (c). In the latter,  all the \ce{C\bond{-} C} bonds are equivalent and the dotted lines are used to represent additional \ce{C\bond{-} C} bonding with a bond order of 1/2 per \ce{C\bond{-} C} pair, delocalised all along the chain.

\begin{figure}[htbp]
\centering
\fbox{\includegraphics[width=\linewidth]{trans_poly}}
\caption{Structure of $trans$-polyacetylene chain : the system is generated by a repeat unit containing 2 carbon and 2 hydrogen atoms and a repeat vector $\vec{a}$. }
\label{fig:trans-poly}
\end{figure}


\begin{figure}[htbp]
\centering
\fbox{\includegraphics[width=\linewidth]{trans_poly2}}
\caption{(a)-(b) Mesomeric Lewis structures for $trans$-polyacetylene;  (c) average structure of $trans$-polyacetylene.}
\label{fig:trans-poly2}
\end{figure}

\begin{figure}[htbp]
\centering
\fbox{\includegraphics[width=\linewidth]{ben_e1}}
\caption{The eigenvalues of the \textbf{H} matrix for benzene. The lowest and the highest energies are non-degenerate while the second and the third energy levels are degenerate. The figure also depicts the filling up of 6 $\pi$ electrons in benzene.}
\label{fig:ben_e1}
\end{figure}


Conjugated molecules, often tend to be planar and hence all the atoms in the molecules can be assumed to lie in the same plane, say the x-y plane. Which means that the molecule will have a reflection symmetry about the z-axis. So we have reflection symmetry about x and y and this gives rise to $\pi_y$ orbitals that are anti-symmetric with respect to reflection and the $\sigma$ orbitals that are symmetric to reflection about the y axis. These $\pi_y$ orbitals will be an linear combination of the $p_y$
orbitals on each of the carbon atoms. 

In general, since the $\pi$ orbitals are the highest occupied orbitals and the $\sigma$ bonds are more strongly bonded -- it makes sense to talk only the $\pi$ bond when talking about bond formation and breaking as implied by the resonance structures shown in Figure \ref{fig:trans-poly2}.  This is a basic approximation of the very simple H$\ddot{\text u}$ckel approach. As always we start with the time-dependent Schrödinger equation,

\begin{align}
 \hat{H}\Psi = E\Psi
 \label{eq:sc}
\end{align}
where $\hat{H}$ is the Hamiltonian and $\Psi$ is the wavefunction and E is the total energy. The next step is to expand the wave function as linear combination of the basis functions,
\begin{align}
 \Psi=\sum_{i=i..n}c_i \chi_i
 \label{eq:scc}
\end{align}
Here the $c_i$ are the expansion coefficients and  $ \chi_i$ are the basis functions, and n the total number of basis functions used. By substituting, Equation  \ref{eq:scc} into Equation  \ref{eq:sc} and applying the variational theorem, we derive the Schrödinger equation in secular from as shown below:
\begin{align}
 det[\textbf{H}-E\textbf{S}]=0.
 \label{eq:sec}
\end{align}
Where \textbf{H} is the matrix whose elements -- called the energy integral-- are given by :
\begin{align}
 H_{ij}=\int \chi_i ^{*} \hat{H}  \chi_j d\tau
 \label{eq:genH}
\end{align}
and \textbf{S}  is the matrix whose elements called the overlap integrals- -are given by,
\begin{align}
 S_{ij}=\int \chi_i ^{*} \chi_j d\tau
 \label{eq:overlap}
\end{align}
In the H$\ddot{\text u}$ckel approach we choose the basis function to be the atomic orbitals. So we choose the $p_y$ orbitals in our case and we apply the first approximation that the orbitals are orthonormal. Which reduces the \textbf{S}  matrix to a identity matrix-- and our generalized eigenvalue problem is reduced to a normal eigenvalues problem as shown in Equation \ref{eq:sc}. The second approximation is to assume that any Hamiltonian integrals vanish if they involve atoms i,j that are not the nearest neighbors. This makes sense as the  $p_y$ orbitals are far apart and have a very limited partial overlap. Also the diagonal terms are the same since these involve the average energy of an electron in a carbon $p_y$ orbital.
\begin{align}
 H_{ii}=\int p_y ^{i} \hat{H}  p_y ^{i} d\tau =\alpha
 \label{eq:pHii}
\end{align}
$\alpha$ is generally called the on-site energy. For any two nearest neighbors, the matrix element is also assumed to be constant. 
\begin{align}
 H_{ij}=\int p_y ^{i} \hat{H}  p_y ^{j} d\tau =\beta
 \label{eq:pHij}
\end{align}
$\beta$ is called the resonance integral and this assumption holds good as long as the  \ce{C\bond{-} C} bond is the remains the same.\\
After the above assumptions as made the \textbf{H} matrix for a molecule say -benzene with 6 carbon atoms looks something like:
\begin{align}
\textbf{H}=
\begin{bmatrix}
    \alpha & \beta & 0 & 0 & 0 & \beta\\
    \beta & \alpha & \beta & 0 & 0 & 0\\
    0 & \beta & \alpha & \beta & 0 & 0\\
    0 & 0 & \beta & \alpha & \beta & 0\\
    0 & 0 & 0 & \beta & \alpha & \beta\\
    \beta & 0 & 0 & 0 & \beta & \alpha
\end{bmatrix}
 \label{eq:pHmat}
\end{align}

Solving,i.e, diagonalizing the \textbf{H} matrix, we find 4 distinct energies as shown in the Figure \ref{fig:ben_e1}. Again, since there are 3 $pi$ bonds in benzene we can fill the  6 electrons by doubly occupying the first three molecular orbitals as shown in the Figure \ref{fig:ben_e1}. The H$\ddot{\text u}$ckel energy of benzene is then calculated as:
\begin{align}
E=2E_1+2E_2+2E_3= 6\alpha+8\beta
 \label{eq:ee}
\end{align}
The eigenvectors corresponding to these energy states can be obtained and they are:
\begin{align}
\textbf{c}^1=\frac{1}{\sqrt{6}}
\begin{bmatrix}
    +1\\
    +1\\
    +1\\
    +1\\
    +1\\
    +1\\   
\end{bmatrix}
\textbf{c}^2=\frac{1}{\sqrt{12}}
\begin{bmatrix}
    +1\\
    +2\\
    +1\\
    -1\\
    -2\\
    -1\\   
\end{bmatrix}
\textbf{c}^3=\frac{1}{\sqrt{4}}
\begin{bmatrix}
    +1\\
    0\\
    -1\\
    -1\\
    0\\
    +1\\   
\end{bmatrix}\\
\textbf{c}^4=\frac{1}{\sqrt{4}}
\begin{bmatrix}
    +1\\
    0\\
    -1\\
    +1\\
    0\\
    -1\\   
\end{bmatrix}
\textbf{c}^5=\frac{1}{\sqrt{12}}
\begin{bmatrix}
    +1\\
    -2\\
    +1\\
    +1\\
    -2\\
    +1\\   
\end{bmatrix}
\textbf{c}^1=\frac{1}{\sqrt{6}}
\begin{bmatrix}
    +1\\
    -1\\
    +1\\
    -1\\
    +1\\
    -1\\   
\end{bmatrix}
 \label{eq:pHmat}
\end{align}



The energy calculated above presents an interesting result -- it predicts that benzene is more stable than a normal 3 double bond system. If we do  H$\ddot{\text u}$ckel theory calculation for etylene molecule we find that as single etylene molecule we find that a single ethylene molecule has an energy of:
\begin{align}
E_{C=C}=2\alpha+2\beta
 \label{eq:ethy}
\end{align}

Thus, for three double bonds this becomes 
\begin{align}
E= 3 \times E_{C=C}=6\alpha+6\beta
 \label{eq:sta}
 \end{align}

This is off by $2\beta$. And since $\beta$  is a measure of the strength of the bonding interaction as a result of the overlap of orbitals i and j orbitals - it is negative for constructive overlap of orbitals - so the $\pi$ electrons in benzene are more stable than a collection of three double bonds. This is called \textit{aromatic stabilization}. 
Now we analyze the bond order. In the  H$\ddot{\text u}$ckel theory the bond order can be defined as:

\begin{align}
O_{ij} =\sum_{\mu=1}^{occ}=c_i^\mu c_j^\mu 
 \label{eq:bond1}
 \end{align}

This definition incorporates the idea that if molecular orbital $\mu$ has a bond between the ith and jth carbons, then the coefficients of the molecular orbital on those carbons should both have the same sign (e.g we have $p_y^{i}+p_y^{j}$), similarly if its an anti-bonding orbital then the coefficients have the opposite sign  (e.g we have $p_y^{i}+p_y^{j}$). The expression above reflects that :
\begin{align}
& c_i^\mu c_j^\mu >0 \, \, \textbf{if} \, \, c_i^\mu c_j^\mu  \,\,\textbf{ have the same sign} \\
& c_i^\mu c_j^\mu <0 \, \, \textbf{if}  \, \,c_i^\mu c_j^\mu  \,\, \textbf{have the opposite sign}
 \label{eq:bond2}
 \end{align} 
 
Equation \ref{eq:bond1} gives a positive contribution for bonding orbitals and a
negative contribution for anti-bonding. A doubly occupied orbital appears twice in the summation.  The summation over the occupied orbitals sums up the bonding or anti-bonding contributions from all the occupied molecular orbitals for the particular i-j pair of carbons to give  the total bond order.  Applying the Equation \ref{eq:bond1} to the $C_1$ and $C_2$ of the benzene molecule we get:

\begin{align}
O_{12}=& 2 c_1^{\mu=1} c_2^{\mu=1}+2 c_1^{\mu=2} c_2^{\mu=2}+2 c_1^{\mu=3} c_2^{\mu=3}
\label{eq:bond21}
 \end{align}
 
\begin{align*}
& = 2 \left( \frac{+1}{\sqrt{6}} \right) \left( \frac{+1}{\sqrt{6}} \right)+
2 \left( \frac{+1}{\sqrt{12}} \right) \left( \frac{+1}{\sqrt{12}} \right)+ \\
& 2 \left( \frac{+1}{\sqrt{4}} \right) \left( \frac{0}{\sqrt{4}} \right)\\
& = 2\frac{1}{6}+2\frac{2}{12}\\
& =\frac{2}{3}
 \label{eq:bondz}
 \end{align*} 

Thus, the $C_1$ and $C_2$ of the benzene share $\frac{2}{3}$ of a $\pi$ bond. Repeating the procedure for each of the $C-C$ atoms we find the same bond order for all of them (taking note of the fact that we have omitted the $\sigma$ orbital contributions). This is again an interesting results, since for a double bond we have a bond order of $\frac{1}{2}$ for each   bond \ce{C\bond{-} C}  $\pi$-bond rather than $\frac{2}{3}$. The additional $\frac{1}{6}$ of a bond per carbon comes directly from the aromatic stabilization -- as the molecule is more stable than a three isolated $\pi$ bonds by $2\beta$. This higher bond order results in a shorter \ce{C\bond{-} C} bond length in benzene as compared to non-aromatic conjugated systems.

This procedure was computationally implemented in the  C++ programming language for benzene, anthracene,  hexacene,  1. 2-benzopentacene, anthanthrene and  peropyrene. The input connectivity files for these molecules are attached with the email and the resulting orbital energies assuming $\alpha= $, $\beta= $ and $\alpha= $, $\beta= $ are also shown below.





\section{Examples of Article Components}


\label{sec:examples}



Table \ref{tab:shape-functions} shows an example table. 

\begin{table}[htbp]
\centering
\caption{\bf Shape Functions for Quadratic Line Elements}
\begin{tabular}{ccc}
\hline
local node & $\{N\}_m$ & $\{\Phi_i\}_m$ $(i=x,y,z)$ \\
\hline
$m = 1$ & $L_1(2L_1-1)$ & $\Phi_{i1}$ \\
$m = 2$ & $L_2(2L_2-1)$ & $\Phi_{i2}$ \\
$m = 3$ & $L_3=4L_1L_2$ & $\Phi_{i3}$ \\
\hline
\end{tabular}
  \label{tab:shape-functions}
\end{table}


\section{Figures and Tables}

It is not necessary to place figures and tables at the back of the manuscript. Figures and tables should be sized as they are to appear in the final article. Do not include a separate list of figure captions and table titles.

Figures and Tables should be labelled and referenced in the standard way using the \verb|\label{}| and \verb|\ref{}| commands.

\subsection{Sample Figure}

Figure \ref{fig:false-color} shows an example figure.

\begin{figure}[htbp]
\centering
\fbox{\includegraphics[width=\linewidth]{sample}}
\caption{False-color image, where each pixel is assigned to one of seven reference spectra.}
\label{fig:false-color}
\end{figure}

\subsection{Sample Table}

Table \ref{tab:shape-functions} shows an example table. 

\begin{table}[htbp]
\centering
\caption{\bf Shape Functions for Quadratic Line Elements}
\begin{tabular}{ccc}
\hline
local node & $\{N\}_m$ & $\{\Phi_i\}_m$ $(i=x,y,z)$ \\
\hline
$m = 1$ & $L_1(2L_1-1)$ & $\Phi_{i1}$ \\
$m = 2$ & $L_2(2L_2-1)$ & $\Phi_{i2}$ \\
$m = 3$ & $L_3=4L_1L_2$ & $\Phi_{i3}$ \\
\hline
\end{tabular}
  \label{tab:shape-functions}
\end{table}





\section{Sample Equation}

Let $X_1, X_2, \ldots, X_n$ be a sequence of independent and identically distributed random variables with $\text{E}[X_i] = \mu$ and $\text{Var}[X_i] = \sigma^2 < \infty$, and let
\begin{equation}
S_n = \frac{X_1 + X_2 + \cdots + X_n}{n}
      = \frac{1}{n}\sum_{i}^{n} X_i
\label{eq:refname1}
\end{equation}
denote their mean. Then as $n$ approaches infinity, the random variables $\sqrt{n}(S_n - \mu)$ converge in distribution to a normal $\mathcal{N}(0, \sigma^2)$.

\section{Sample Algorithm}

Algorithms can be included using the commands as shown in algorithm \ref{alg:euclid}.

\begin{algorithm}
\caption{Euclid’s algorithm}\label{alg:euclid}
\begin{algorithmic}[1]
\Procedure{Euclid}{$a,b$}\Comment{The g.c.d. of a and b}
\State $r\gets a\bmod b$
\While{$r\not=0$}\Comment{We have the answer if r is 0}
\State $a\gets b$
\State $b\gets r$
\State $r\gets a\bmod b$
\EndWhile\label{euclidendwhile}
\State \textbf{return} $b$\Comment{The gcd is b}
\EndProcedure
\end{algorithmic}
\end{algorithm}

\section{Results}


\section{Discussions}


\section*{Funding Information}
Graduate Teaching Assistantship by the Chemistry Department at the University of Illinois at Urbana-Champaign (UIUC). Computational resources were provided by the Hirata Lab, UIUC.
\section*{Acknowledgments}

Many thanks to Prof. So Hirata for his guidance and patience while teaching this course

%\bigskip \noindent See \href{link}{Supplement 1} for supporting content.



% Bibliography
\bibliography{sample}

% Full bibliography added automatically for Optics Letters submissions
% Note that this extra page will not count against page length
\ifthenelse{\boolean{shortarticle}}{%
\clearpage
\bibliographyfullrefs{sample}
}{}

%Manual citation list
\begin{thebibliography}{1}
\bibitem{wiki1}
Karl, S.  \textit{J. Chem. Educ.}, 2013, 90 (4), 463–469.
\bibitem{wiki2}
Senn, P. \textit{ J. Chem. Educ.}, 1992, 69 (10), 819-821.
\bibitem{wiki3}
Bacci, M.  \textit{J. Chem. Educ.}, 1982, 59 (10), 816-818.
\bibitem{wiki4}
https://en.wikipedia.org/wiki/H\%C3\%BCckelmethod accessed on March 23rd, 2017.
\bibitem{wiki5}
https://ocw.mit.edu/courses/chemistry/5-61-physical-chemistry-fall-2007/lecture-notes/lecture31.pdf accessed on March 23rd, 2017.
\bibitem{wiki6}
Levine, I. N. \textit{Quantum Chemistry}, 5th ed.; Prentice Hall:: Upper Saddle River, NJ, 1999; p 739.
\bibitem{wiki7}
  Szabo, A. , Ostlund N. S. \textit{  Modern Quantum Chemistry: Introduction to Advanced Electronic Structure Theory}, Dover Books on Chemistry::31 East 2nd Street, Mineola, NY 11501-3852


\end{thebibliography}

\end{document}
